\documentclass[11pt,fleqn,a4paper]{article}

\usepackage[a4paper,bottom=0cm]{geometry}
\usepackage{url}
\usepackage{amsmath}
\setlength{\voffset}{-0.75in}
\setlength{\headsep}{5pt}

\newcommand{\half}{\frac{1}{2}}
\newcommand{\mainEquation}{\( \sum\limits_{i=1}^{n} i = \half n(n+1) \quad 1 \leq n < \infty  \)}

\begin{document}
\title{Proof that \mainEquation}
\date{}
\maketitle

\section{Introduction}
\[ S \text{ is a set such that } n \in S \text{ if } \sum\limits_{i=1}^{n} i = \half n(n+1) \]
\[ \text{By proving that } n \in S \text{ for } i \leq n < \infty \]
 \[ \text{it will prove that \mainEquation } \]

\section{The Induction Axiom}
The fifth of Peano's axioms, which states: If a set \(S\) of numbers contains zero and also the successor of every number in \(S\), then every number is in \(S\). \footnote{from \url{http://mathworld.wolfram.com/InductionAxiom.html}}

Zero refers to the first element of the set, which in this case is 1. 

\section{Proof that \(1 \in S \)}
\[ \sum\limits_{i=1}^{1}i = 1 \]
\[ \half (1)(1 + 1) = 1 \]

\section{Proof that if \(n \in S \) then \( n + 1 \in S \)}
\[ \sum\limits_{i=1}^{n + 1}i = \left(\sum\limits_{i=1}^{n}i\right) + n + 1 \]
\[ f:\rightarrow f(n) = \half n(n+1) \]
\[ f(n + 1) = f(n) + n + 1\]
\[ \half (n + 1)(n + 1 + 1) = \half n(n + 1) + n + 1 \]
\[  \half (n^2 + 3n + 2) = \half (n^2 + n) + n + 1) \]
\[ \frac{n^2}{2} + \frac{3n}{2} + 1 = \frac{n^2}{2} + \frac{n}{2} + n + 1 \]
\end{document}